\section{Linux 账号管理与 ACL 权限设置}

\subsection{用户账号}
系统处理流程,
\begin{itemize}
    \item 进入 /etc/passwd中寻找输入的账号,读取 UID 以及对应的 GID(/etc/group%
    中),若未找到则跳出;

    \item 进入/etc/shadow找出对应的账号和UID,核对密码;

    \item 以上都通过进入则进入shell.
\end{itemize}

相关文件结构,
\begin{itemize}
    \item /etc/passwd文件结构:\\
    例如, \verb|yong:x:1000:1000:Yong:/home/yong:/bin/bash|
    \begin{enumerate}
        \item 账号名称:yong;
        \item 密码:该字段经过加密保存于/etc/shadow中;
        \item UID:User ID;
        \item GID:Group ID;
        \item 用户信息说明列;
        \item 主文件夹:/home/yong;
        \item Shell:bash.
    \end{enumerate}

    \item /etc/shadow文件结构:\\
    例如, \verb|yong:$6$0...(省略):16310:0:99999:7:::|
    \begin{enumerate}
        \item 账号名称:yong;
        \item 密码:经过加密的字符串;
        \item 最近更改密码的日期:从1970年1月1日作为1开始累加;
        \item 密码不可被更改的天数:0,表示可以随意改动;
        \item 密码需要重新更改的天数:99999,不强制你更改之意;
        \item 密码需要更改期限前的警告天数:7天;
        \item 密码过期后的账号宽限时间(密码失效日):空;
        \item 账号失效日期:空;
        \item 保留:空.
    \end{enumerate}
\end{itemize}

\subsection{用户组}
相关文件结构,
\begin{itemize}
    \item /etc/group文件结构:\\
    记录GID与组名的对应关系,例如, \verb|family:x:1002:yong,luo|
    \begin{enumerate}
        \item 用户组名称:family;
        \item 密码:该字段经过加密保存于 /etc/gshadow中;
        \item GID:Group ID;
        \item 此用户组支持的账号名称:逗号隔开,不含空格.
    \end{enumerate}

    \item /etc/gshadow文件结构:\\
    例如, \verb|family:!:yong:yong,luo|
    \begin{enumerate}
        \item 用户组名称:family;
        \item 密码:``!"表示无密码;
        \item 管理员账户:yong;
        \item 用户组所属账户:与 /etc/group内容相同.
    \end{enumerate}
\end{itemize}

有效用户组:创建文件默认所属的用户组(\texttt{groups} 命令查看第一个即是,使用%
命令\texttt{newgrp} 更改).

\subsection{账号管理}
\subsubsection{创建账号}
\begin{itemize}
\renewcommand{\arraystretch}{1.1}
    \item \texttt{useradd [参数选项] 用户账户名 }\footnote{在Debian中,默认不会创建主文件夹,且shell默认为/bin/dash, 因此最好手动指定这些参数.}
    \begin{longtable}{c@{: }p{0.8\columnwidth}}\hline\hline
      \multicolumn{2}{l@{}}{\bfseries 参数:}\\
     \texttt{-u } & 直接指定 UID;\\
     \texttt{-g } & 初始用户组;\\
     \texttt{-G } & 次要用户组;\\
     \texttt{-M } & 强制!不创建主文件夹(系统账号默认);\\
     \texttt{-m } & 强制!创建主文件夹;\\
     \texttt{-c } &  /etc/passwd第5列说明内容,可以随便设置;\\
     \texttt{-d } & 直接指定主文件夹(使用绝对路径),而不使用默认值;\\
     \texttt{-r } & 创建系统账号;\\
     \texttt{-s } &  shell,若不指定,默认为/bin/bash;\\
     \texttt{-e } & 日期, ``YYYY-MM-DD",账号失效日期,写入 /etc/shadow;\\
     \texttt{-f } & 密码是否失效,0为立即失效, -1为永远不失效,shadow第7个字段.\\\hline
    \end{longtable}

     \item \texttt{useradd} 参考文件数据是 /etc/default/useradd\footnote[1]{Debian中,默认的许多项处于注释状态,且shell使用的是/bin/sh, 指向/bin/dash.}, 例如,
     \begin{longtable}{l@{: }p{0.6\columnwidth}}\hline\hline
        \texttt{HOME=/home} & 用户主文件夹基目录;\\
        \texttt{INACTIVE=-1} & 密码过期后是否失效;\\
        \texttt{EXPIRE=} & 账号失效日期;\\
        \texttt{SHELL=/bin/bash} & 默认使用的shell程序文件名;\\
        \texttt{CREATE\_MAIL\_SPOOL=yes} & 创建用户的mailbox;\\
        \texttt{SKEL=/etc/skel} & 用户主文件夹参考基准目录,新建用户主文件各项数据都是有该目录赋值过去的.\\\hline
    \end{longtable}

     \item UID/GID以及密码参数参考文件是 /etc/login.defs,可以使用%
     命令\\ \verb<cat /etc/login.defs | sed -r '/#.*$|^$/ d' | sort< \\%
     查看该文件默认设置.
\end{itemize}

总结:使用 \texttt{useradd} 程序创建账号时,会参考 /etc/default/useradd、%
/etc/login.defs、\\
/etc/skel/* 并向 /etc/passwd、/etc/shadow、/etc/group、/etc/gshadow写入数据,%
以及创建用户主文件夹.系统账号shell使用/sbin/nologin,故无法登录shell,可以新建%
\textbf{/etc/nologin.txt} 文件,试图登录系统账号时,屏幕出现该文件内容.

\begin{itemize}
\renewcommand{\arraystretch}{1.1}
\item 与创建账号相关的命令
\begin{longtable}{l@{: }p{0.8\columnwidth}}\hline\hline
     \texttt{passwd} & 设置密码的相关参数\\

     \texttt{chage} & 显示以及修改/etc/shadow文件密码的相关字段,例如可要求账号首次登陆修改密码\\

     \texttt{usermod} &  /etc/passwd和/etc/shadow与账号相关数据的微调\\

     \texttt{userdel} & 删除用户的相关数据.若只是暂停该账号使用,可以将%
    /etc/shadow账号失效日期(第8列)设置为0即可.\\\hline
\end{longtable}

\item 用户功能的相关命令
\begin{longtable}{l@{: }p{0.8\columnwidth}}\hline\hline
     \texttt{finger} & 查阅很多用户的相关信息\\

     \texttt{chfn} & 有点像change finger,修改一些可查阅的信息,例如办公室号码\\

     \texttt{chsh} &  change shell\\

     \texttt{id} & 查阅账号的UID/GID等信息.\\\hline
\end{longtable}


\end{itemize}

\subsubsection{创建用户组}
\begin{itemize}
    \item \texttt{groupadd [-g GID] [-r]} 用户组名

    \item \texttt{groupmod [-g GID] [-n group\_name]} 用户组名:修改GID或组名;

    \item \texttt{groupdel}:删除用户组;

    \item \texttt{gpasswd}:用户组管理员功能,设置账号为管理员,将用户添加进用户组或从用户组中删除用户.

    \begin{itemize}
        \item \texttt{gpasswd groupname}:设置密码;
        \item \texttt{gpasswd [-A user1,...] [-M userA,...] groupname}: A 为设置管理员,M为添加用户进组;
        \item \texttt{gpasswd [-rR] groupname}: r为删除密码,R为密码栏失效;

        \item \texttt{gpasswd [-ad] user groupname}:管理员添加与删除用户.
    \end{itemize}
\end{itemize}

\subsection{ACL}
ACL可以针对单一用户或用户组、单一文件或目录进行权限设置.
\begin{itemize}
\renewcommand{\arraystretch}{1.1}
    \item \texttt{setfacl [-bkRd] [{-m|-x} acl 参数] 目标文件名}
    \begin{longtable}{c@{: }p{0.8\columnwidth}}\hline\hline
      \multicolumn{2}{l@{}}{\bfseries 参数:}\\
    \texttt{-m} & 设置ACL参数;\\
    \texttt{-x} & 删除ACL参数;\\
    \texttt{-b} & 删除所有的ACL参数;\\
    \texttt{-k} & 删除默认的ACL参数;\\
    \texttt{-R} & 递归设置ACL参数;\\
    \texttt{-d} & 对目录设置默认ACL参数,在该目录新建数据将引用此值.\\\hline
    \end{longtable}

    \begin{enumerate}
        \item \texttt{u:[用户账号列表]:[rwx]}:对特定用户设置\\
        \texttt{setfacl -m u:yong:rwx aclDir}

        \item \texttt{g:[用户组列表]:[rwx]}:对用户组设置\\
        \texttt{setfacl -m g:family:rx aclDir}

        \item \texttt{m:[rwx]}:设置mask有效权限,用户与用户组实际权限不超过此设置\\
        \texttt{setfacl -m m:rwx aclDir}

        \item \texttt{d:[ug]:用户列表:[rwx]}:设置默认权限,用户创建的文件都默认该ACL设置\\
        \texttt{setfacl -m d:u:yong:rx aclDir}
    \end{enumerate}

    \item \texttt{getfacl filename}:一个文件设置ACL参数以后,它的权限部分多出一个 \texttt{+} 号,使用该命令可以查询详细的ACL 参数.
\end{itemize}

\subsection{切换身份}
\begin{itemize}
\renewcommand{\arraystretch}{1.1}

    \item \texttt{su [-lm] [-c 命令] [username]}
    \begin{longtable}{c@{: }p{0.8\columnwidth}}\hline\hline
      \multicolumn{2}{l@{}}{\bfseries 参数:}\\
    \texttt{-} & 使用login-shell的变量文件读取方式登录系统,不加用户名表示切换为root;\\
    \texttt{-l} & 与上类似,但需要加用户名;\\
    \texttt{-m} & 与\texttt{-p}一样,使用目前的环境设置,而不读取新用户的配置文件;\\
    \texttt{-c} & 仅进行一次命令.\\\hline
    \end{longtable}

    \item \texttt{sudo [-b] [-u 新用户账号]}\\
    该命令会在/etc/sudoers文件中查找用户是否具有执行sudo的权限,该文件使用%
    \texttt{visudo}来编辑.通过修改该文件,\textbf{可以限定用户或用户组可执行的命令,可%
    免除密码等}.
\end{itemize} 